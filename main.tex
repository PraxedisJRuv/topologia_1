\documentclass[12pt]{article}
\usepackage[spanish]{babel}
\usepackage{amssymb}
\usepackage{amsmath}
%Falta agregar una portada bien hecha aparte.
\LARGE{\title{Tareas de primer parcial-Topología}}
\author{Alumnos: \\Arturo Rodriguez Contreras - 2132880 \\
Jonathan Raymundo Torres Cardenas - 1949731\\
Praxedis Jimenes Ruvalcaba \\
Erick Román Montemayor Treviño - 1957959 \\
Alexis Noe Mora Leyva \\}
\begin{document}
\maketitle
%Intenten usar este formato para comenzar un problema
\paragraph{1}
\textit{¿Es la unión de topologías una topología?}

Sea $X = \{ a, b, c \}$, $\tau_{1} = \{\emptyset, X, \{a\} \}$, $ \tau_{2} = \{\emptyset, X, \{b\} \}$
se tiene que $\tau_{1}$ es topología ya que contiene al conjunto vacio, $X$, contiene las uniones arbitrarias
$\{a\} \cup X = X \in \tau_{1}$, y tambien
$\{a\} \cup \emptyset = \{a\} \in \tau_{1}$
y $\emptyset \cup X = X \in \tau_{1}$
y contiene a las intersecciones finitas de sus elementos, 
de igual forma se sigue que $\tau_{2}$ es topologia de $X$. La union de las dos topologías es
$U = \tau_{1} \cup \tau_{2} = \{ \emptyset, X, \{a\}, \{b\} \}$
lo cual no es topolog\'ia, ya que $\{a\} \cup \{b\} = \{a, b\} \notin U$, por lo tanto, no
necesariamente la unión de topologías es una topología.\\

\paragraph{2}
\textit{Demostrar que $\tau_{\mathbb{N}}$ es topologia.}

Se tiene por definicion que $\{\emptyset, X\} \subset \tau_{\mathbb{N}}$. Ahora, sea $\{U_{a}\}_{a \in J}$ una coleccion de elementos en $\tau_{\mathbb{N}}$,
y $U = \bigcup_{a \in J} U_{a}$. Queremos ver que $U \in \tau_{\mathbb{N}}$, para
esto observemos que $X - U = (\bigcup_{a \in J} U_{a})^{c} $ por leyes de De Morgan es igual a
$\bigcap_{a \in J} U_{a}^{c}$, sabemos por teorema quela intersecci\'on arbitraria de conjuntos contables es tambien
contable, entonces $\bigcap_{a \in J} U_{a}^{c} \in \tau_{\mathbb{N}}$ paraya que $U_{a}^{c} = X - U_{a}$ es por definicion contable
para todo $a \in J$. Luego, tomemos$\{U_{a}\}_{a \in J}$ una colecci\'on finita de elementos en
$\tau_{\mathbb{N}}$, y sea $U = \bigcap_{a \in J}U_{a}$ entonces tenemos $X - U = (\bigcap_{a \in J}U_{a})^{c}$ por leyes de DeMorgan es igual a $\bigcup_{a \in J}U_{a}^{c}$ y por teorema
la union finita de conjuntos contables es tambien
contable. Entonces $X - U$ es contable, por lo cual se tiene que $U \in \tau_{\mathbb{N}}$ entonces $\tau_{\mathbb{N}}$ esta
cerrado por intersecci\'on finita, como consequente es una topolog\'ia.\\

\paragraph{3}
\textit{Verificar si $\tau_{\infty}$ es topologia.} 

Sea $X=\mathbb{R}$, sea $U_1 = (-\infty,0),U_2 = (0,\infty)$, claramente $U_1,U_{2}\hspace{2px}\in\hspace{2px}\tau_{\mathbb{N}}$, pero $U=U_1\cup U_2 = (\infty,0)\cup(0,\infty)\notin \tau_{\infty}$
ya que $\mathbb{R}-U=\{0\}$ no es infinito. Por lo tanto no cumple el axioma de uniones arbitrarias de topología.\\ $\therefore\tau_{\infty}$ no es topología.\\

\paragraph{4}
\textit{Demostrar que $(0,1)=\bigcup\limits_{n\in\mathbb{N}-\{1\}}[\frac{1}{n},1)$.}

$(\subset)$

Sea $x\in(0,1)$, esto es $0<x<1$ , por propiedad arquimediana existe $ N \in\mathbb{N}\hspace{4px}t.q \hspace{4px} \forall n\geq N \hspace{4px} 0< \frac{1}{n} \leqslant  x<1$ entonces $x\in[\frac{1}{n},1)$
entonces $x\in\bigcup\limits_{n\in\mathbb{N}-\{1\}}[\frac{1}{n},1)$\\$\therefore(0,1)\subset\bigcup\limits_{n\in\mathbb{N}-\{1\}}[\frac{1}{n},1)$

$(\supset)$

Sea $x\in\bigcup\limits_{n\in\mathbb{N}-\{1\}}[\frac{1}{n},1)$, entonces $0<\frac{1}{n_0}\leq x<1$ para algun $n_0\in\mathbb{N}-\{1\}$\\$x\in(0,1)$\\$\therefore(0,1)\supset\bigcup\limits_{n\in\mathbb{N}-\{1\}}[\frac{1}{n},1)$

\paragraph{5}
\textit{Verificar que $\beta_K$ satisface el teorema de creación de topologías.}
%procedimiento
\paragraph{6}
\textit{Verificar que $B_S=\{(a,b]:a<b$\} es base de algunas topologias}
%procedimiento

\paragraph{7}
\textit{Verificar las comparaciones entre $\tau_S, \tau_L$ y entre $\tau_S, \tau_K$}

\paragraph{8}
\textit{Demostrar si $\tau_{\mathbb{R}^2}=\tau_{\mathbb{R} \times \mathbb{R}}$}
\\Primero vemos si \( \tau_{\mathbb{R}^2}\) es más fina que \(\tau_{\mathbb{R} \times \mathbb{R}}\)
\\Ahora verificamos la otra contención.
\\Sea \(x\in(a,b)\)x\((c,d): a<b y c<d\)
\\tomamos \(\delta=min\{(x,(a,b)),(x,(c,d)),(x,(a,c)),(x,(b,d))\}\)
\\entonces, La bola \((x,\frac{\delta}{2})\subset(a,b)\)x\((c,d)\)
\\donde \(x\in(x,\frac{\delta}{2})\)
\\\(\therefore\tau_{\mathbb{R} \times \mathbb{R}}\subset\tau_{\mathbb{R}^2}\)
\paragraph{9}
\textit{Terminar paso inductivo del teorema de las proyecciones}

\paragraph{10}
\textit{Demuestra que $(A \times B) \cap (C \times D)= (A \cap C) \times (B \cap D)$.}

$(\subset)$

Sea $(x,y) \in (A \times B) \cap (C \times D)$. Por definición de intersección $(x,y)\in A\times B$ y $(x,y) \in C\times D$. Además, por definición
de producto cruz $x \in A $ y $y \in B$, $x \in C$ y $y \in D$. Reescribiendo obtenemos $x\in A$ y $x\in C$, $y \in B$ y $y\in D$ \textit{i.e.}  $(x,y) \in (A \cap C)\times (B \cap D)$.

$(\supset)$

De forma análoga, sea $(x,y) \in (A \cap C) \times (B \times D)$ luego  $x \in A $ y $y \in B$, $x \in C$ y $y \in D$ y $(x,y) \in (A \times B) \cap (C \times D)$.

\paragraph{11}
\textit{Verificar que la topologia del orden $\tau(\beta_O)$ es topología.}

\paragraph{12}
\textit{Verificar si $\tau_O (\mathbb{N})=\tau_d (\mathbb{N})$.}

\paragraph{13}
\textit{Verificar que el orden lexicografico genera un orden en $\mathbb{R}$.}

\paragraph{14}
\textit{Verificarlas comparaciones entre $\tau_O(\mathbb{R})$ y $\tau_{d\times d}$}

\paragraph{15}
\textit{Demuestre que $\overline{A \cap B} \subset \overline{A} \cap \overline{B}$.}

Sabemos que $A \subset \overline{A}$ y $B \subset \overline{B}$. Luego $A \cap B \subset \overline{A} \cap \overline{B}$. La cerradura de un conjunto es siempre
cerrado y la intersección de cerrados es cerrada \textit{i.e.} $A \cap B$ está contenido en un cerrado y la cerradura es el cerrado más pequeño que contiene al conjunto. Es decir
$\overline{A \cap B} \subset \overline{A} \cap \overline{B}$.

\paragraph{16}
\textit{Verificar si $\overline{\bigcup\limits_{\alpha\in J}A_{\alpha}}=\bigcup\limits_{\alpha\in J}\overline{A_{\alpha}}$}

\paragraph{17}
\textit{Verificar si $X-\overline{A}=\overline{X-A}$}

\paragraph{18}
\textit{Considere $([0,1])^2$ bajo $\tau_{\mathbb{R}_L\times\mathbb{R}_S}$ hallar $Int([0,1]^2)$}

\paragraph{19}
\textit{Verificar si $Int(A\cap B)=Int(A)\cap Int(B)$}

\paragraph{20}
\textit{Verificar que si $D_{1}, D_{2}$ son densos y abiertos en $X$, entonces $D_{1} \cap D_{2}$ es denso en $X$}

Sea U un abierto arbitrario de X, tenemos que por
asociatividad de la intersecci\'on
$U \cap (D_{1} \cap D_{2}) = (U \cap D_{1}) \cap D_{2}$,
y $(U \cap D_{1})$ es abierto por la segunda axioma de
topolog\'ia. Ahora bien, tenemos que la intersecci\'on de
cualquier abierto con $D_{2}$ es no vacio ya que $D_{2}$ es
denso, entonces $(U \cap D_{1}) \cap D_{2} \neq \varnothing$.
Juntando todo lo que tenemos, $U \cap (D_{1} \cap D_{2})$
$= (U \cap D_{1}) \cap D_{2} \neq \varnothing$, por teorema
se tiene que $D_{1} \cap D_{2}$ es denso.

\end{document}
